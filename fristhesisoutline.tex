\documentclass[a4paper,12pt, oneside]{article}
\usepackage{geometry} 
\geometry{letterpaper}
\usepackage[parfill]{parskip}
\usepackage{graphicx}
\usepackage{amssymb}
\usepackage{amsmath}
\usepackage[usenames, dvipsnames]{color}
\usepackage{subcaption}
\usepackage [english]{babel}
\usepackage [autostyle, english = american]{csquotes}
\usepackage{float}
\usepackage{verbatim}
\MakeOuterQuote{"}

\newcommand{\red}[1]{\textcolor{red}{#1}}
\newcommand{\blue}[1]{\textcolor{blue}{#1}}
\newcommand{\green}[1]{\textcolor{green}{#1}}


\title{Thesis Outline -- Very Drafty Draft}
\author{Lizzie Hannah}
\date{\today}

\begin{document}
\maketitle
\section{Abstract}
\section{Introduction (i.e. brief history/background/motivation for research)}

\blue{Tom comments in blue.}

\green{Things to remove/change are in green.}

Flying discs have long provided a source of entertainment for dogs and humans alike.  The modern Frisbee -- first produced by Wham-O Inc. toy company in 1957 -- is today a staple in household toy boxes across the United States.  While the precise details of the Frisbee's history continue to be a source of debate, experts generally credit Civil War era baker William Russel Frisbie with designing the baking tin that ultimately became the first Frisbee prototype.  Yale University students purchased Frisbie's pies and played catch by tossing their empty tins around; thus the idea for the Frisbee was borne (1).  

In the late 1940s, Walter Frederick Morrison designed and manufactured the first plastic disc, eventually partnering with Wham-O for mass production and marketing (2). Since then, flying discs have exploded in popularity, providing the basis for sports like Ultimate Frisbee and disc golf. Ultimate Frisbee, a game that combines elements of football and soccer into a fast-paced field sport, is estimated to be played by 7 million people around the globe (4).  Disc golf (as the name suggests) bears many similarities to standard golf: disc golfers compete by throwing discs from "tees" towards targets, aiming to reach the target with as few throws as possible. Though disc sports are relatively young, they are slowly gaining widespread acceptance in the arena of competitive athletics. Indeed, in 2016 the International Olympics Committee granted full recognition to the World Flying Disc Federation, indicating that disc sports may appear in future Olypmic Games (5).

While Wham-O continues to hold a trademark on the term "Frisbee," numerous companies (including Discraft, Innova, EuroDisc, and others) have emerged as competitors in the flying disc market. It is worth mentioning here that, while all flying discs retain the same basic shape, it is possible to observe subtle differences in the trajectories of their flights.  A Frisbee produced by Wham-O, for example, may veer to one direction at the end of a backhand throw, while an Ultrastar produced by Discraft may tail in the opposite direction at the end of a similar throw (3).  As disc sports become increasingly competitive, it will be in the best interest of athletes, coaches, and fans to better understand the aerodynamics of of flying discs.

Despite the recent growth of disc sports, the current body of Frisbee research remains relatively limited. Previous research has developed physics-based models of flying discs, but these models have yet to be refined such that they can predict the flight trajectory of any disc given any set of initial conditions. In particular, the accuracy of the existing models depends heavily on the accuracy of experimental flight data, which is difficult to obtain (2).

As mentioned above, understanding the dynamics of Frisbee flight may enable athletes and coaches to improve their performance in sports like Ultimate Frisbee and disc golf.  More generally, however, spinning flying objects display unique and interesting aerodynamic properties, governed by the same principles that guide sophisticated flying vehicles and spacecraft (6). Simply put, a Frisbee is a combination of a wing and a gyroscope. Thus, Bernoulli's Principle (which controls the motion of wings) provides the lift that keeps a Frisbee in the air during flight, and the spin of a Frisbee provides the gyroscopic stability required for a flying disc to maintain its directionality (9). Because of their size and accessibility, Frisbees are convenient objects on which to study patterns of flight in a lab setting. Thus, developing accurate and reliable models of Frisbee throws may prove relevant in designing more complex flying objects. 
 
To date, the most comprehensive model of a thrown disc is described in \textit{Frisbee Flight Simulation and Throw Biomechanics}, published by Sarah Hummel in 2003. Hummel's thesis provides a nearly exhaustive review of currently available Frisbee literature and offers a model that builds on the work of (among others) Potts \& Crowther, Yasada, Mitchell, and Stilley \& Carstens (2). 

Ultimately, the goals of this Honors Thesis are twofold. The first goal is to reproduce Hummel's model in a format that is accessible to a relatively wide, general audience, and the second is to test Hummel's model against actual flight data obtained from video footage of thrown discs. By reproducing Hummel's work and refining her model where it fails to match reality, we can add to the existing body of Frisbee literature and help broaden the knowledge of flying disc dynamics for future researchers.

\section{Explanation of Forces \& Torques}
Understanding the forces and torques that act upon a disc during flight is essential to modeling the trajectory of a Frisbee throw. The following descriptions ignore environmental factors (like wind and rain) that might affect a disc's flightpath, and they further ignore factors such as airflow that could only be accounted for using fluid dynamics. Instead, the forces and torques outlined here appeal to solely classical mechanics in their description of a Frisbee's motion.

\textit{Forces}

Three primary forces influence the translational motion of a flying disc: lift, drag, and gravity (2). Gravity, of course, points vertically downward from the Frisbee's center of mass.  The drag force points in the direction opposite the disc's velocity, working to slow down the Frisbee, and the lift force acts in the direction perpendicular to drag, \green{usually opposing gravity} [\blue{this depends on the orientation of the disc. 'Usually' is subjective. What if I only threw hammers?}] (2). The total force acting on a Frisbee is the sum of all the forces.
\begin{figure}[h]
        \includegraphics[width=6cm, height=4cm]{frisforces}
	\centering
	\caption{\red{This diagram is from Hubbard and Hummel, 2000 (8); not sure if I'm allowed to use it w/o permission or if I should make a new one.} \blue{Attribution is fine.}}
\end{figure}

To calculate the magnitude of the gravitational force acting on a Frisbee, we simply multiply the mass of a standard Frisbee (0.175 kg) by the constant of gravitational acceleration, here defined as  9.81 $m/s^{2}.$ 
 
The calculation of the lift and drag forces acting on a flying disc, though slightly more complicated than the calculation of the gravitational force, is completed in a similarly standard way. The magnitudes of the lift and drag forces (denoted $F_L$ and $F_D$) are defined by \green{the following equations} [\blue{I actually don't know the best convention for a segway into writing equations. What do you think?}]\green{:}

\begin{equation}
  F_D=C_DAv^2\rho/2,
\end{equation}
\begin{equation}
  F_L=C_LAv^2\rho/2
\end{equation}

where \textit{A} is the area of a standard disc (0.057 $m^2$), \textit{v} is the velocity of a thrown disc at time \textit{t}, and $\rho$ is the density of air (taken here to be the average value at sea level of $1.225 kg/m^3$). $C_L$ and $C_D$ represent the coefficients of lift and drag, respectively. Both $C_L$ and $C_D$ are functions of $\alpha$, the angle of attack, defined as the angle between the disc's velocity and the plane of the disc (Figure 1). 

The lift and drag coefficients further depend on a series of parameters that are specific to individual discs (or types of discs); these parameters are constant values that serve to distinguish the trajectory of one Frisbee throw from that of another.  Experiments (\red{include citations here}) show that the drag coefficient has quadratic dependence on $\alpha$, and the lift coefficient has linear dependence on $\alpha$. We will use $\alpha_0$ to denote the value of $\alpha$ at which $F_D$ is at a minimum, here assumed to be $-4^{\circ}$ (2).

\begin{equation}
  C_D=P_{D0}+P_{D\alpha}(\alpha-\alpha_0)^2
\end{equation}
\begin{equation}
  C_L=P_{L0}+P_{L\alpha}\alpha
\end{equation}
\begin{figure}[H]
	\centering
	\includegraphics[width=8cm, height=5cm]{DragCoefficientPlot}
	\caption{\color{red}Also include lift vs. angle of attack plot once i figure out how to use the subfigure package. And also make better plots. \color{black}}
\end{figure}
\color{BurntOrange}
It is these parameters ($P_{D0}, P_{D\alpha}, P_{L0}$, and $P_{L\alpha}$) that form the crux of this Honors Thesis. Previous research has attempted to calculate $P_{D0}, P_{D\alpha}, P_{L0}$, and $P_{L\alpha}$, but to date there are discrepancies within the literature regarding the true parameter values (Figure 3). By making small modifications to Hummel's model and comparing the model's output to data taken from actual Frisbee flights, the work presented here will attempt to calculate the value of each individual drag and lift parameter for a \red{Discraft Ultrastar} disc.

\begin{figure}[H]
	\centering
	\includegraphics[width=8cm, height=5cm]{ParameterValues}
	\caption{\color{red}Top line is values reported by Hummel. Figure from Hummel Thesis \color{black}}
\end{figure}
\textit{Torques}

In order to produce an accurate model of a Frisbee flight in three dimensions, we must refer not only to the forces that act on a flying disc but also to the torques that cause the disc to rotate about each of its axes. As with the lift and drag forces, the magnitudes of the torques in the \textit{x}, \textit{y}, and \textit{z} directions are calculated using three standard equations that depend on a series of parameters (2). Here we will use $\tau$ to denote torque.\begin{comment} and we will define $\phi$ to be the angle about the \textit{x}-axis, $\theta$ to be the angle about the \textit{y}-axis, and $\gamma$ to be the angle about the \textit{z}-axis.
\begin{figure}[H]
	\centering
	\includegraphics[width=8cm, height=5cm]{AxesDiagram}
	\caption{\red{Fix and make prettier, obvi}}
\end{figure}
\end{comment}

The magnitude of the torque around the \textit{x}-axis depends on two parameters, $P_{\tau_x\omega_x}$ and $P_{\tau_x\omega_z}$, which are multiplied by $\omega_x$ and $\omega_z$, respectively. From here on $\omega$ denotes angular velocity. According to Hummel, Potts \& Crowther, and others,
\begin{equation}
  \tau_x=(P_{\tau_x\omega_x}\omega_x+P_{\tau_x\omega_z}\omega_z)\frac{1}2Av^2d.
\end{equation}
As before, \textit{A} is the area of a standard disc in $\textit{m}^2$, \textit{v} is the disc's velocity at time \textit{t}, and \textit{d} is the diameter of a standard disc, here taken to be $\sqrt\frac{1.14}\pi$ \textit{m}.

Similarly $\tau_y$ and $\tau_z$ are written as,
\begin{equation}
  \tau_y=(P_{\tau_{y0}}+P_{\tau_y\omega_y}\omega_y+P_{\tau_{y\alpha}}\alpha)\frac{1}2Av^2d
\end{equation}
\begin{equation}
  \tau_z=(P_{\tau_z\omega_z}\omega_z)\frac{1}2Av^2d.
\end{equation}

\color{black}
\section{Explanation of Newton's Equations of Motion} 

According to classical mechanics, the motion of rigid bodies can be described by two of Newton's equations, which relate an object's translational and rotational trajectory to the sum of the forces and torques acting upon it (7).  The Newton-Euler equations of motion are derived from Euler's two laws of motion for rigid bodies:
\begin{equation}
\vec{F}=\textit{m}\vec{a}
\end{equation}
\begin{equation}
\vec{M}=\dfrac{\vec{dL}}{dt}
\end{equation}
Equation 1 states that the sum of the forces acting upon a rigid body is equal to the product of the body's mass $(\textit{m})$ and its acceleration $(\vec{a})$, where $\vec{a}$ is simply a time derivative of the body's velocity.  

Equation 2 states that a rigid body's angular momentum $(\vec{dL}/dt)$ changes at a rate equal to the sum of the torques $(\vec{M})$ acting upon it. Here we have $\vec{L}$=\textit{I}$\vec{w}$, where \textit{I} is the mass moment of inertia matrix and $\vec{w}$ is the body's angular velocity vector. Due to the axial symmetry of a disc, the inertial matrix for a Frisbee is defined as,  
\begin{equation*}
I=\begin{bmatrix}
I_{xx} & 0 & 0 \\
0 & I_{yy} & 0 \\ 
0 & 0 & I_{zz}
\end{bmatrix},
\end{equation*}
where $I_{xx}$=$I_{yy}$ and $I_{zz}$ is distinct (8).  The nature of $\vec{w}$ depends on a series of rotations that rotate the Frisbee between a standard \textit{xyz} coordinate axis to a new axis, denoted \textit{x'y'z'} (2). These rotations are discussed in the following sections.

\textit{Euler Rotations} \newline
\red{Note: I feel like I should cite Tom here, or at least the textbook he used to summarize this info for me, Goldstein, Poole, \& Safko}
\blue{That's fine. Let me know if you want section numbers to cite specifically.}

\blue{It is important to note somewhere that Hummel defines the +$z$ direction to be down, while you define it to be up.}

Hummel's Frisbee model takes into account several reference frames, which are coordinate systems that describe an object's orientation.  In the following discussion, the term "lab frame" refers to an inertial \textit{x, y, z} reference frame, where the \textit{xy} plane is horizontal and the positive \textit{z} axis points upward.  The term "frisbee frame" refers to a body-fixed coordinate system, \textit{x', y', z'}, in which the \textit{x'y'} plane lies on top of -- or parallel to -- the plane of the Frisbee.

We rotate the coordinates of a disc between the \textit{x,y,z} and \textit{x',y',z',} axes by performing a series of matrix rotations about intermediate sets of axes.  The rotations used here are those described in Hummel (2003), which rotate the Frisbee as follows: 1) about the \textit{x}-axis through angle $\phi$ into an intermediate set of axes denoted $\chi,\eta,\zeta$, 2) about the $\eta$-axis through angle $\theta$ into an intermediate set of axes denoted $\chi',\eta',\zeta'$, and 3) about the $\zeta'$ axis through angle $\gamma$ into \textit{x', y', z'}.

Rotations 1-3 are achieved with three rotation matrices, which can be multiplied together to form a single comprehensive rotation matrix. Rotation 1 can be expressed by the matrix,
\begin{equation*}
C(\phi)=\begin{bmatrix}
1 & 0 & 0 \\
0 & \cos\phi & \sin\phi \\
0 & -\sin\phi & \cos\phi
\end{bmatrix}.
\end{equation*}

Rotation 2 can be expressed by the matrix, 
\begin{equation*}
B(\theta)=\begin{bmatrix}
\cos\theta & 0 & -\sin\theta \\
0 & 1 & 0 \\
\sin\theta & 0 & \cos\theta
\end{bmatrix}.
\end{equation*}

Rotation 3 can be expressed by the matrix, 
\begin{equation*}
D(\gamma)=\begin{bmatrix}
\cos\gamma & \sin\gamma & 0 \\
-\sin\gamma & \cos\gamma & 0 \\
0 & 0 & 1
\end{bmatrix}.
\end{equation*}

In general, we would express the final Euler rotation matrix as a combination of B, C, and D. However, for the purposes of modeling a Frisbee trajectory, we have chosen to rotate the original \textit{x, y, z} axis simply onto the plane of the disc, not onto the spinning disc. This choice was made for the sake of simplifying the model and the numerical calculations it requires (2). Rotating the \textit{x, y, z} axis onto the plane of the disc only requires rotations 1 and 2, which rotate through $\phi$ and $\theta$, respectively.

Therefore, the complete rotation matrix is described by the following:

\begin{equation*}
A(\phi,\theta,\gamma)=A(\phi,\theta)=B(\theta)C(\phi)=\begin{bmatrix}
\cos\theta & \sin\phi\sin\theta & -\sin\theta\cos\phi \\
0 & \cos\phi & \sin\phi \\
\sin\theta & -\sin\phi\cos\theta & \cos\phi\cos\theta
\end{bmatrix}.
\end{equation*}

\section{Angular Velocities}
We can denote the angular velocities acting on the Frisbee as $\vec{w}_\phi$, $\vec{w}_\theta$, and $\vec{w}_\gamma$.  According to Goldstein, Poole, \& Safko, these angular velocities point along the "axes that their angles are rotated about."

Consider, for example, $\vec{w}_\phi$.  Before performing any rotations, the vector $(\dot\phi, 0, 0)$ points along the original \textit{x}-axis.  The original \textit{x}-axis is ultimately rotated through angles $\phi$ and $\theta$ via matrices C and B, as described above. In other words, the original \textit{x}-axis undergoes a full transformation by the matrix A$(\phi,\theta)$. Therefore, in order to determine the direction of $\vec{w}_\phi$, we must rotate $(\dot\phi, 0, 0)$ about angles $\phi$ and $\theta$.  This rotation yields:

[\blue{Remember to punctuate at the end of an equation if it is the end of a sentence.}]
\begin{equation*}
\vec{w}_\phi=A(\phi,\theta)\left(\begin{array}{ccc}\dot\phi\\0\\0\end{array} \right)=\left(\begin{array}{ccc}\dot\phi\cos\theta\\0\\\dot\phi\sin\theta\end{array} \right).
\end{equation*}

Similarly, in order to find $\vec{w}_\theta$ and $\vec{w}_\gamma$, we can rotate $(0, \dot\theta, 0)$ and $(0, 0, \dot\gamma)$ about the angles through which their original axes are rotated. The vector $(0, \dot\theta, 0)$ originally points along the intermediate $\eta$ axis, which only rotates through the angle $\theta$. As such, we have:

[\blue{Sometimes to make these linear algebra equations easier to read I do this (See the .tex file).}]
\begin{equation*}
  \vec{w}_\theta=B(\theta)\left(\begin{array}{ccc}
    0\\
    \dot\theta \\
    0
  \end{array} 
  \right)=\left(\begin{array}{ccc}
    0 \\ 
    \dot\theta \\
    0
  \end{array}\right).
\end{equation*}

Since we have only chosen to rotate the original axes of the Frisbee onto the plane of the disc (ignoring spin), the rate at which the angles in the Frisbee frame are changing is equal to the sum $\vec{w}_\phi+\vec{w}_\theta$. Thus,
\begin{equation*}
\vec{w}_F=\left(\begin{array}{ccc}\dot\phi\cos\theta\\\dot\theta\\\dot\phi\sin\theta\end{array} \right)
\end{equation*}

The angular velocity of the Frisbee itself accounts for the final component, $\vec{w}_\gamma$, which already lies along the \textit{z'}-axis after undergoing rotations through $\phi$ and $\theta$. This means that the vector $(0, 0, \dot\gamma)$ does not require any additional rotations, and 
\begin{equation*}
\vec{w}_{Total}=\vec{w}_\gamma+\vec{w}_F=\left(\begin{array}{ccc}\dot\phi\cos\theta\\\dot\theta\\\dot\phi\sin\theta+\gamma\end{array} \right)
\end{equation*}
\section{Second Derivative Equations}

Ultimately, grouping Euler's laws of motion together in terms of vectors and matrices yields the Newton-Euler equations of motion (expressed below), which are assumed to govern the flight paths of Frisbees (Hummel 2003). 
\begin{equation}
\vec{F}=\textit{m}(\dfrac{\vec{dv}}{dt}+\vec{\textit{w}}_F\times\vec{v})
\end{equation}
\begin{equation}
\vec{M}=I\dfrac{\vec{dw}}{dt}+\vec{\textit{w}}_F\times I \vec{w}
\end{equation}

\color{BurntOrange}
Since we have already written explicit equations for $\textit{w}_F$ and $\textit{w}_{Total}$, we can write the righthand side of Equations 9 and 10 in terms of $\phi$, $\theta$, and $\gamma$. In doing so, we can derive \red{differential equations} that describe both the translational and angular accelerations of a Frisbee throughout its flight. These equations can be numerically integrated in order to determine the position and velocity of a disc at any time during its trajectory. 

First, we calculate $\vec{w}_F\times\vec{v}$ as follows: 
\begin{equation*}
\vec{w}_F\times\vec{v}=\begin{bmatrix}
w_x & w_y & w_z \\
v_x & v_y & v_z \\
\hat{x} & \hat{y} & \hat{z}
\end{bmatrix}
\end{equation*}

\begin{equation*}
=\left(\begin{array}{ccc} v_z\theta'-v_y(\gamma'+\phi'\sin\theta) \\ v_x(\gamma'+\phi'\sin\theta)-v_z\phi'\cos\theta \\ v_y\phi'\cos\theta-v_x\theta'\end{array}\right)
\end{equation*}

Substituting $\vec{w}_F\times\vec{v}$ into (9), we can solve for acceleration. Thus,

\begin{equation}
\frac{{dv}_x}{dt}=\frac{{F}_x-\theta'v_z+v_y(\phi'\sin\theta)}{m}
\end{equation}

\begin{equation}
\frac{{dv}_y}{dt}=\frac{F_y-v_x\phi'\sin\theta+v_z\phi'\cos\theta}{m}
\end{equation}

\begin{equation}
\frac{{dv}_z}{dt}=\frac{F_z-v_y\phi'\cos\theta+v_x\theta'}{m}
\end{equation}

Next, to solve for angular acceleration, we observe that 
\begin{equation*}
\frac{\vec{dw}}{dt}=\left(\begin{array}{ccc}\ddot\phi\cos\theta-\dot\phi\dot\theta\sin\theta\\ \ddot\theta \\ \ddot\phi\sin\theta + \dot\phi\dot\theta\cos\theta+\ddot\gamma\end{array} \right)
\end{equation*}

Substituting $\frac{\vec{dw}}{dt}$ into (10) and solving for $\ddot\phi$, $\ddot\theta$, and $\ddot\gamma$ yields,
\begin{equation}
\ddot\phi=\frac{M_x-I_{zz}\dot\theta(\dot\phi\sin\theta+\dot\gamma)+2I_{xx}\dot\phi\dot\theta\sin\theta}{I_{xx}\cos\theta}
\end{equation}

\begin{equation}
\ddot\theta=\frac{M_y+I_{zz}\dot\phi\cos\theta(\dot\phi\sin\theta+\dot\gamma)-I_{yy}\dot\phi^2\sin\theta\cos\theta} {I_{yy}}
\end{equation}

\begin{equation}
\ddot\gamma=\frac{M_z-I_{zz}(\dot\phi\dot\theta\cos\theta-\ddot\phi\sin\theta)}{I_{zz}}
\end{equation}

\color{black}
\section{Works Cited (Tentative)}

1. http://www.wfdf.org/history-stats/history-of-fyling-disc/4-history-of-the-frisbee
2. Hummel thesis \newline
3. Should I cite some sort of observation here? [\blue{Yes, potts and crowther and the others}]\newline
4. http://www.usaultimate.org/about/\newline
5. http://www.wfdf.org/news-media/news/press/2-official-communication/697-international-olympic-committee-grants-full-recognition-to-the-world-flying-disc-federation-wfdf
6. Lorenz 2005 -- Flight and attitude dynamics measurements of an instrumented Frisbee 
\end{document}  
7. Cite a classical mechanics textbook (i.e. NOT wikipedia)
8. Hubbard and Hummel (2000) -- Simulation of Frisbee Flight
9. V.R. Morrison -- The Physics of Frisbees
